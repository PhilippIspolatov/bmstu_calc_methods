% !TEX root = main.tex

\section{Постановка задачи}

\subsection{Содержательная постановка задачи}

Пусть есть n работ и n исполнителей. Стоимость выпонения i-й работы j-м исполнителем 
составляет $C_(ij)$ >= 0. Требуется распределить все работы так, чтобы:

\begin{itemize}
    \item каждый исполнитель выполнял одну работу;
    \item общая стоимость выполнения всех работ была минимальной/максимальной;
\end{itemize}

\subsection{Математическая постановка задачи}

\begin{equation*}
    \begin{cases}
        f = \sum_{i=1}^{n}\sum_{j=1}^{n}x_{ij}c_{ij} \rightarrow min\\      
        \sum_{j=1}^{n}x_{ij} = 1, i=\overline{1,n}\\
        \sum_{i=1}^{n}x_{ij} = 1, j=\overline{1,n}\\
        x_{ij} \in {0,1}
    \end{cases},
    \text{где f - целевая функция, x - матрица назначений}
\end{equation*}